\documentclass[sigconf]{acmart}
\usepackage[utf8]{inputenc}
\usepackage[round]{natbib}


\begin{document}
\title{Standards for smart cities}
\author{Rahul Thiruthinmel Premnavas}
\date{December 2022}
\email{Rahul.ThiruthinmelPremnavas@student.hs-rm.de}


\affiliation{%
  \institution{RheinMain University of Applied Sciences}
  \streetaddress{P.O. 65197}
  \city{Wiesbaden}
  \state{Hesse}
  \country{Germany}
  \postcode{43017-6221}
}




\begin{abstract}
  Smart cities are the ultra developed urban areas which are enhanced by the information and communication technology (ICT) and hardware components connected to Internet of Things (IoT) to bring the best of the efficiency in the cities (service, management, wellness of the citizens). However these smart cities, being rigorously pursued and corrected by developers and innovators, are always under scrutiny. Thus resulted in setting up of standards. Most often those are evaluated by the criteria of national and international standards. In this paper we would be predominantly looking at the technical standards associated with the smart city conventions and would be also looking into the structure of Smart Cities, it's components, some pilot projects.
\end{abstract}
\keywords{Smart city, information and communication technology, internet of things, international standards, technical standards.}

\maketitle

\section{Introduction}
The percentage of population in the cities is nowhere getting down, let alone for the next 20 to 30 years\cite{lai2020review}. Obviously that creates shortages of resources in various spheres i.e. energy, water resources, electricity and most importantly social and environmental sustainability. The latter,i.e.environmental sustainability, is the most challenging part faced by modern societies \cite{lai2020review}. As per TWI Global \cite{twiglobal} "the concept of smart cities began as far back as the 1960's and 1970's when the US Community Analysis Bureau began using databases, aerial photography and cluster analysis to collect data, direct resources and issue reports in order to direct services, mitigate against disasters and reduce poverty". As we discussed earlier the idea of smart city and the technologies behind these spectrum's became cogent. Information Technology plays a very important role in shaping up smart cities by incorporating the most modern technologies such as solutions provided by artificial intelligence and machine learning, most advanced sensors, etc. Having said that, the governments and societies are responsible to address the challenges which comes along with it. According to Gandy and Nemorin \cite{gandy2019toward} the major three challenges are:
\begin{enumerate}
\item{commodification}: treating people in the societies/systems more or less as commodities) 
\item{social and digital exclusion} \item{privacy and surveillance}
\end{enumerate}

\section{Various Domains of a Smart City(as per Deloitte)}
	Moreover, the functioning of a smart city is the incorporation of various domains \cite{gandy2019toward}. According to Deloitte they can be divided into six key domains\cite{lai2020review}.
 \subsection{Energy and Environment}
Technologies are very much used and to make sustainable growth in the urban areas, according to \cite{fedorova2019balancing} sensors can be used to monitor leakages, plant growth etc, thus by supporting citizens to makes decisions in resource utilization. The environment is suffering at the moment and to measure the damage caused to it by the over utilization of resources and by many other doings, there is no other good way than to implement the smart city concepts which help measure, analyse and re-implement the practises thus getting the optimal results, which to many extend reduce the damage caused by the environment decay.

 \subsection{Economy}
 According to \cite{lai2020review}, by implementing technologies in the economic domain, various aspects associated with it would drastically change and the need for new skill sets would also arise. Thus with the new technology driven market every smart cities would be the power houses of generation of wealth. The demand of high skilled labours and intuition into new technologies would be a demanded custom. The new innovations thus could enhance the development and thus creating the new revolution in industry, coined as industry version 4.0. 
  \subsection{Safety and Security}
 For every population/society it is of the utmost importance to encrypt their private data and thus making it sure, that it is  preserved and not liable to breach. The risk of data breach and exploitation is always there, especially when one is implementing a new technology. The officials in the area of data protection are implementing the most modern cryptographic tools to make sure the breaching does not occur, and this tools can be implemented in various other domains as well. This domain is crucial to make the system stay afloat.
 \subsection{Health and Living}
Hospitals represent a busy environment with the majority of the medical operations relying on electricity, for example lighting and medical equipment's. Hence, it is important to deliver high quality of services, improve patient's well being as well as to take measures for the reduction of the hospital’s carbon footprint and the it's other harmful impacts on the environment \cite{taha2018reduction}. The study made by Taha, Ahmed, Wu, Ruiheng and Emeakaoroha \cite{taha2018reduction} showed that how technology driven remedial measures reduce the costs of electricity. Efficient implementation of technologies in smart cities can thus make an all round improvement in this domain.
\subsection{Mobility}
The modern mobility consists of automated vehicles and different types of shared mobility services which obtained through Internet of Things \cite{lai2020review}. As per \cite{guerrero2015integration} IoT is the collaboration of devices communicating with each other instead of people and thus will take over the internet communication. A study conducted by Yin, Su and Xun \cite{yin2020data} showed that advanced data driven approaches helped to improve the train control algorithms. 
\subsection{Education and Government}
The data analytics which are available can be used to carve out the best plans for the educational sector and for the schemes and functioning of the government. There by the outcomes of the due year can be predicted and plotted by different ITC (Information Technology and Communication) concepts. Smart learning environments (SLEs) as discussed in \citep{hoel2018standards} lead to stable, coherent and exhaustive understanding, thereby providing standards development for learning, education and training a needed direction.
\section{Various other Domains}
Similarly \citep{giffinger2007smart} Giffinger, Christian, Hans see smart cities  having six domains which are pretty much in resonance with Deloitte, namely \textbf{smart economy, people, governance, mobility, living}.
Apart from these six domains there can be some others too, such as \textbf{smart water, smart health, smart waste} \citep{lai2020review}. A smart city is literally a big changeover for all the aspects of a region, with the application of technology. Its indeed mandatory to regulate the standards of the development process, methods and practices.
\section{International Standards}
 As per International Organization of Standardization standard is “the first step towards the holy grail of an inter operable, plug-and-play world where cities can mix and match solutions from different vendors without fear of lock-in or obsolescence or dead initiatives" \citep{whataresmartcities}. The standards are thus intended to bring out the best results while implementing any technology or new methods. 
The well established standards are the International Standards Organization (ISO) \citep{whataresmartcities}, the International Electrotechnical Commission (IEC) \citep{IEC}, the International Telecommunication Union (ITU) \citep{ITU} and Institute of Electrical and Electronics Engineers (IEEE). A brief description of those will be handy:
\subsection{IEC}
More than 173 countries as partners and presenting more than 10,000 standards IEC promotes through its members the cooperation in electro-technical standards in fields of electricity and related technologies \citep{IEC}. According to IEC \cite{IEC} "standards are way of doing things by repetition for the best of results". IEC has the luxury of standardisation management boards which takes care of the standardisation work. In collaboration with ISO, for example, the ISO/IEC24668 (a joint technical committee of ISO and IEC): Process Management for Big Data Analytics provides a reliable framework for acquiring, processing, storing data irrespective of any sector it operates. Thus enabling the organisations to work on big data to get practices in Artificial Intelligence. Notable is also the ISO/IEC 27001 standard which provides essential cyber security guidance to the data. Information security management system described in ISO/IEC 27001 is known as the most trusted cyber security framework \citep{editorial}. Environment conscious design (ECD) is one of the branches in IEC which cannot be overlooked. Clearly based on sustainable development goals, it's aims are to develop sustainable products, to minimise waste and facilitate recycling \citep{editorial}. Materials are reused, refurbished and recycled to make the sustainability principles congealed with the ideals. A detailed description of respective standards can be found in IEC 62430, one among the many standards related to the circular economy in IEC (IEC 63430: Environmentally conscious design, it's principles, requirements and guidance \citep{editorial})


\subsection{ITU}
It's global membership include 193 member states and 900 companies, universities etc. They are founded to facilitate international connectivity, they allocate global radios spectrum and satellite orbits. Every time when we makes a call via mobile, connect internet or send an email, we benefit from the work of ITU.
United for Smart Sustainable cities (U4SSC) is a global UN initiative coordinated by ITU and supported by various other bodies of UN to achieve sustainable goals \citep{ITU}. The set of ICT (Information and Communication Technology) Key Performance Indicators in following areas like infrastructure, education, households and businesses helps the cities/countries to do the information collection and guides to improve the statistics and thus helps to improve the standards \cite{ITU}.
\subsection{ISO}
It covers almost all aspects of technology and manufacturing. ISO has quality management standards to help to work more efficiently and reduce product failures, the environmental management standards to help to reduce environmental impacts, also to reduce waste and to become more sustainable, health and efficiency standards to help reduce accidents in the workplace, energy management standards to help cut energy consumption, food safety standards to help prevent food being contaminated, IT security standards to keep sensitive information secure \citep{whataresmartcities}
Of all the ISO/IEC 27001:2022 the joint work of ISO and IEC (as already discussed in the previous paragraph (IEC)) tutelages the requirements for establishing, implementing, maintaining and continually improving an information security management system \citep{whataresmartcities} and deals with the quality management systems and the requirements. For example to meet the needs of the customers with the product and services, the enhancement of customer satisfaction by the improving the system and by applying statutory regulations.


\par Responsible use of resources, preserving the environment and improving the well-being of citizens are the end goal for experts of ISO Technical Committee ISO/TC, sustainable cities and communities, whose flagship standard ISO 37101 helps communities define their sustainable development objectives and implement strategies to achieve them \citep{whataresmartcities}.
In ISO 37120:2018 (indicators for city services and quality of life) they have came up with indicators for measuring and monitoring progress of a city and they have standardized indicators to measure the performance of city services, prioritize the solutions and thus improve the quality for life. The whole structure pedestal of these indicators and later then divided to two other sets namely ISO 37122 (smart cities) and ISO 37123 (resilient cities). The indicators are used by ISO are:
\subsubsection{\textbf{Core Indicators}}
Indicators that are required to demonstrate performance in the delivery of city services and quality of life \citep{whataresmartcities}.
\subsubsection{\textbf{Supporting Indicators}}
They can be city objectives and they are used to demonstrate performance in delivery and quality of life\citep{whataresmartcities}
\subsubsection{\textbf{Profile Indicators}}
Profile indicators are informative references and recommended to provide basic statistics and background information \citep{whataresmartcities}.

\section{additional information on IEC and ITU }
\par IEC brings together more than 173 countries and provides global, neutral and independent standardization platform to 20000 experts globally as discussed earlier. Moreover it administers 4 conformity assessment systems whose members certify that devices, systems, installations, services and people work as required \citep{IEC}. Given that the majority of the energy is being used in the cities, IEC can play a major role in checking the usage according to it's standards in order to bring about the best model in building smart cities. IEC also help connect and automate city infrastructure \citep{IEC}. Infrastructures and their interconnection is something which needs a lot of scrutiny, precision work and they all are shattered around with different suppliers, but the IEC helps to coordinate and standardise all of them to safely connect and automate all of them. IEC contributes also into smart electrification and electric generation, transportation etc.
\par ITU via its study groups, adopts the information and communication technologies in smart and sustainable cities \citep{sang2019itu}. The standards published by the ITU deals with the terminologies in IoT's, the interoperability of ICT's and IoT's networks. The whole frame work of the SSC (Smart Sustainable Cities) standards can be classified into four.
\begin{enumerate}
\item SSC management and assessment standards.
\item SSC service standards
\item Standards – includes Data layer standards, communication layer standards etc.
\item Building and physical infrastructure standards.
\end{enumerate}

Much of the work of ITU is carried out by its study groups as discussed earlier, each containing experts in respective fields.
\subsection{IEEE}
Institute of Electrical and Electronics Engineers is also a well known organization in setting standards for smart cities. It consists of 71 published standards and good 50 odd under development \cite{IEEE}.
In 2017, the IEEE P2784 the smart city planning guide was propose to develop a framework that mentions the processes and technologies for planning the smart city transformation \cite{lai2020review}. Some of the most recent initiatives are listed below.
\subsubsection{\textbf{IEEE standards in development of smart grid and smart energy}}
In the year 2016 IEEE released standards namely 1889-2018 to evaluate and test the electrical performance of Energy saving devices \citep{lai2020review}. The scope of these include as name suggested the analysis of the performances of the energy devices. They also can be used to test the load with connected power saving devices and then later without those devices to get the right measurements. They provide detailed protocols for the testing circuits, their accuracy and the details of instrumentation and sequence of the test measurements. These standards can be used in all kinds of electrically connected energy saving devices.
The other standard P1922.1 \cite{P1922.1} formulates the methods to compute anticipated emissions created by virtual machine migration and allocation in distributed locations created by various electricity sources. It also make techniques to study about emissions (also greenhouse gases) and other particle emissions created by virtual machine migration. P2814 \citep{P2814} sets the standards for techno economic metrics for operation and development of electrical energy systems and renewable energy systems. And finally the P2852 \citep{P2852} uses the artificial intelligence technology to get the accurate locations of infrastructure and surroundings.Furthermore it also helps to evaluate the health hazard risks associated.
\subsubsection{\textbf{IEEE standards in development of smart health}}
As per \cite{lai2020review} P3333.2.5 establishes the Bio CAD File Format for medial Dimensional Printing which helps to make 3D prints from the sectional scanning data and thus aids in getting better results in health/surgery sector and P1752 (for mobile health data) states requirements for mobile health data applications and its standardised representations. Finally P2621.1 sets the framework for the electronic product security \cite{IEEE}.
\subsubsection{\textbf{IEEE standards in development for smart mobility and transportation's}}
As per \citep{lai2020review} P1884 has its scope in methods, principles in engineering design, commissioning and installation in rail transit systems. Electrical and Electro Mechanical Bench Test Equipment for Transit Rail Projects are included in P1883 and P2406 states the design and construction of Non Load Break Disconnect Switches on Transit Systems. And finally P2020 deals with the key elements of image and quality (Automotive System Image Quality) \cite{IEEE}.

\subsubsection{\textbf{IEEE Standards in development for smart education}}
Again as per \citep{P7919.1} P7919.1 arranges and explains the e-Readers to support Learning applications and support open source. The secure and trusted learning systems, i.e., P2834 \cite{P2834} tells in detail the importance of privacy protection, 1589-2020 tells the augmented reality learning experience model \cite{lai2020review}, which develops an wide range integrated theoretical model to connect the activities in the current world. In a nutshell it can be said that the standards do play a vital roll in the establishment of smart systems, where the commitment and shared experience of many comes into picture. 



\section{Pilot Projects}
\par Dubai – Aims primarily the wellness of it's citizens Dubai Smart city initiative, comprises more than 130 initiatives by the government and the private sector, include Dubai AI Road Map, Block Chain technology, the Happiness Agenda \cite{khan2017smart.

\par 
Japan – Japan smart community alliance consists of businesses from sectors like manufacturing, electricity, gas, heat supply etc. OF all of it the commendable is the Woven City, an ecosystem which is fuelled by the hydrogen fuel cells which at the the Mount Fiji since 2021 \cite{ToyotaWovenCity}.
\par
Romania- In Romania the Romanian Association for Smart city \cite{theRomanianAssociationforSmartCity} carries out pilot projects, being supported by more than 200 national and international partners. They have also incorporated ISO standards in legislation. The Pilot Project Alba Iulia also promotes collaboration in sectors like education, government, research, other associations etc \cite{lai2020review}.

\section{Conclusion}
Smart cities are subjected to change partly by the ever evolving standards. The existing challenges for the smart cities are points for contemplation and opens up a wide scope of research. The domains of Smart cities are also broad and varying. The entries of new smart city domains into the existing group wouldn't be surprising. This paper can be handy for a brief overview of different types of existing standards, and their parent organisations. The Pilot Projects in Dubai, Japan and Romania are the manifestation of incorporation of various endeavors, including standardisation. The paper serves as a guide for the overview for seeing Smartcity as an entity and Standards as it serving dependencies.





\bibliographystyle{plainnat}
\bibliography{smartcity}





\end{document}
